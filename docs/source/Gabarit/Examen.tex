\def\SigleNumero{MAT-1906}
\def\NomCours{Géométrie pour l'enseignement au préscolaire/primaire}
\def\DateEvaluation{16 juin 2022 de 12h30 à 15h20}
\def\Session{Été 2022}
\def\NomEvaluation{Examen 3}
\def\Ponderation{35}
\def\Enseignante{}				% Lorsque le cours n'est pas à section multiple, indiquez le nom de l'enseignante
\def\Enseignant{Jérôme Soucy}	% Lorsque le cours n'est pas à section multiple, indiquez le nom de l'enseignant
\def\Coordonnatrice{}			% Ne rien mettre entre les accolades si l'enseignant ou l'enseignante coordonne le cours
\def\Coordonnateur{}			% Ne rien mettre entre les accolades si l'enseignant ou l'enseignante coordonne le cours
\def\SectionA{}					% Ne rien mettre entre les accolades s'il s'agit d'un cours à section unique
\def\SectionB{}					% Mettre le nom de chacun des enseignants et enseignantes entre les accolades
\def\SectionC{}					% Une case à cocher sera générée pour chacune des sections (dès qu'il y en a 2)
\def\SectionD{}
\def\PapierLegal{}				% Écrire quelque chose entre les accolades pour imprimer sur du papier de formal légal
								% Le formal légal n'est pas recommandé pour les cours du SSE. Parfois les surveillants
								% n'ont pas accès à ce format de papier.
\def\Directives{
	\begin{enumerate}
		\item Assurez-vous que cette \'evaluation comporte \numquestions ~questions réparties sur \numpages~pages. Elle sera notée sur \numpoints.
		\item Le verso des feuilles peut être utilisé comme brouillon, \textbf{mais il ne sera pas corrigé}.
		\item La dernière page de l'examen peut être utilisée comme brouillon ou pour compléter une solution. Par exemple, si vous souhaitez y poursuivre la solution de la question 1, \textbf{indiquez dans la zone de réponse de la question 1 que la suite se trouve à la dernière page}.
		\item Aucune page de cet examen (y compris la page \numpages) ne doit être débrochée.
		\item Assurez-vous d'identifier correctement votre copie en plus d'inscrire vos initiales au bas des pages 2 à \numpages.
		\item Veuillez déposer une carte d'identité avec photo sur le coin du bureau où vous rédigez l'examen.
		\item Sauf indication contraire, vous devez justifier votre raisonnement.
		\item Une feuille de résultats choisis sera distribuée avec l'examen.
		\item Tout autre document est interdit.
		\item Une calculatrice de base est autorisée. On entend par là une calculatrice avec des fonctionnalités comparables ou inférieures à la Casio FX-55 Plus.
		\item Dans tous les cas où c'est possible, vous devez écrire la valeur exacte et non une valeur numérique approchée (par exemple, si $x^2=2$ et $x>0$, vous devez écrire $x=\sqrt{2}$ plutôt que  $x\approx 1,414$).
	\end{enumerate}
	}
\documentclass{Gradescope}  % On précise la classe utilisée
\usepackage{libertine}      % Facultatif : utilisation de la police Libertine
\usepackage[output-decimal-marker={,}]{siunitx}          % Facultatif : paquet pour gérer les unités
%\usepackage{commandesJS}    % Facultatif : Commandes personnalisées dans le fichier commandesJS.sty

\begin{document}
\begin{questions}
\question[4] Un rectangle a des dimensions de \SI{1.2}{\centi\meter} par \SI{333}{\milli\meter}.
             Trouvez son aire (exacte) en \SI{}{\centi\meter\squared}. 
             Réponse: \boite{2cm}{1cm} \SI{}{\centi\meter\squared}

\question[1\half] Noircissez le carré correspondant à l'isométrie que vous appréciez le plus.
		\begin{checkboxes}
			\choice $r_{O,-170^{\circ}}\circ g_{a,\overrightarrow{UV}}$
			\choice $r_{O,170^{\circ}}\circ g_{a,\overrightarrow{VU}}$
			\choice $r_{O,-170^{\circ}}\circ g_{a,\overrightarrow{VU}}$
			\choice $g_{a,-\overrightarrow{UV}}\circ r_{O,-170^{\circ}}$
			\choice $g_{a,\overrightarrow{UV}}\circ r_{O,170^{\circ}}$
			\choice $g_{a,\overrightarrow{VU}}\circ r_{O,170^{\circ}}$
		\end{checkboxes}

\question Ceci est une question avec des sous-questions.
    \begin{parts}
    \part[1] Première sous-question.
    \vspace{3cm}
    \part[3] Deuxième sous-question.
    \vspace{3cm}
    \end{parts}

\question[3] Ceci est une question nécessitant une longue démarche. Il faut trouver un polynôme $p(x)$ respectant certaines conditions. Écrivez votre réponse dans l'encadré au bas de la page.
\vfill
Réponse: $p(x)=$ \boite{4cm}{1cm}
\end{questions}

\newpage
	Vous pouvez poursuivre une solution sur cette page ou l'utiliser comme brouillon. Dans tous les cas, veuillez ne pas la débrocher. Si vous poursuivez une solution, indiquez dans la zone de réponse de la question concernée que la suite se trouve ici.
\end{document}
